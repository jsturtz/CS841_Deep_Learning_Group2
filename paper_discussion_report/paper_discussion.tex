
% NOTES: Could have a section talking explicitly about the concept of the "Memory Cell"
% COuld have a section at the end that discusses choices for model parameters in a specific example
% Could have section in beginning to go over the basics of recurrent neural networks

\documentclass[journal]{IEEEtran}
\usepackage{tikz}
\usepackage{lipsum}
\usetikzlibrary{decorations.pathreplacing,angles,quotes}

% correct bad hyphenation here
\hyphenation{op-tical net-works semi-conduc-tor}

\begin{document}
%
% paper title
\title{Paper Discussion Report}

% puts info about authors at bottom of first page
\author{Lawrence~Owusu,~Jordan~Sturtz,~and~Swetha~Chittam~% <-this % stops a space
  \thanks{The authors are graduate students at NCA\&T}% <-this % stops a space
}

% note the % following the last \IEEEmembership and also \thanks -
% these prevent an unwanted space from occurring between the last author name
% and the end of the author line. i.e., if you had this:
%
% \author{....lastname \thanks{...} \thanks{...} }
%                     ^------------^------------^----Do not want these spaces!

% The paper headers
% The only time the second header will appear is for the odd numbered pages
% after the title page when using the twoside option.
\markboth{CS851 - Deep Learning: Paper Discussion Report}%
{Shell \MakeLowercase{\textit{et al.}}: CS851 - Deep Learning: Method Discussion Report}

% make the title area
\maketitle

% As a general rule, do not put math, special symbols or citations
% in the abstract or keywords.
% \begin{abstract}
% The abstract goes here.
% \end{abstract}

% Note that keywords are not normally used for peerreview papers.
% \begin{IEEEkeywords}
% IEEE, IEEEtran, journal, \LaTeX, paper, template.
% \end{IEEEkeywords}

% For peer review papers, you can put extra information on the cover
% page as needed:
% \ifCLASSOPTIONpeerreview
% \begin{center} \bfseries EDICS Category: 3-BBND \end{center}
% \fi

%
% For peerreview papers, this IEEEtran command inserts a page break and
% creates the second title. It will be ignored for other modes.
\IEEEpeerreviewmaketitle

\section{Introduction}

%
% Here we have the typical use of a "T" for an initial drop letter
% and "HIS" in caps to complete the first word.
\IEEEPARstart{I}{introduction} goes here.

\section{Problem Statement}
\lipsum[1] % Generate latin nonsense

\section{Related Work}
\lipsum[1] % Generate latin nonsense

\subsection{Current Results on Proposed Problem}
\lipsum[1] % Generate latin nonsense

\section{Data Collection}
\lipsum[1] % Generate latin nonsense

\section{Data Preprocessing}
\lipsum[1] % Generate latin nonsense

\section{Proposed Model}
\lipsum[1] % Generate latin nonsense

\subsection{Model Architecture}
\lipsum[1] % Generate latin nonsense

\subsection{Model Parameters}
\lipsum[1] % Generate latin nonsense

\section{Paper Results}
\lipsum[1] % Generate latin nonsense

\subsection{Comparison Results}
\lipsum[1] % Generate latin nonsense

\section{Conclusion}
\lipsum[1] % Generate latin nonsense

\begin{thebibliography}{1}

\bibitem{IEEEhowto:kopka}
X.~Lastname1 and X.~Lastname2, "Title of Article", \emph{Journal Name}, vol. X, no. Y, pp. XXX-YYY, 2022.
\end{thebibliography}

\end{document}
